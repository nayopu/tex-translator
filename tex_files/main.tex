% \documentclass[xelatex,ja=standard,jafont=noto]{bxjsarticle}

%%%% ijcai20-multiauthor.tex

\typeout{Test}

% These are the instructions for authors for IJCAI-20.

\documentclass{article}

\usepackage[whole]{bxcjkjatype}
\pdfpagewidth=8.5in
\pdfpageheight=11in
% The file ijcai20.sty is NOT the same than previous years'
\usepackage{ijcai20}
% Use the postscript times font!
\usepackage{times}
\renewcommand*\ttdefault{txtt}
\usepackage{soul}
\usepackage{url}
\usepackage[hidelinks]{hyperref}
\usepackage[utf8]{inputenc}
\usepackage[small]{caption}
\usepackage{graphicx}
\usepackage{amsmath}
\usepackage{amssymb}
\usepackage{nccmath} 
\usepackage{bbm}
\usepackage{amsfonts}
\usepackage{booktabs}
\usepackage{todonotes}
\usepackage{algorithm}
\usepackage{algorithmic}
\urlstyle{same}

% the following package is optional:
%\usepackage{latexsym} 

% Following comment is from ijcai97-submit.tex:
% The preparation of these files was supported by Schlumberger Palo Alto
% Research, AT\&T Bell Laboratories, and Morgan Kaufmann Publishers.
% Shirley Jowell, of Morgan Kaufmann Publishers, and Peter F.
% Patel-Schneider, of AT\&T Bell Laboratories collaborated on their
% preparation.

% These instructions can be modified and used in other conferences as long
% as credit to the authors and supporting agencies is retained, this notice
% is not changed, and further modification or reuse is not restricted.
% Neither Shirley Jowell nor Peter F. Patel-Schneider can be listed as
% contacts for providing assistance without their prior permission.

% To use for other conferences, change references to files and the
% conference appropriate and use other authors, contacts, publishers, and
% organizations.
% Also change the deadline and address for returning papers and the length and
% page charge instructions.
% Put where the files are available in the appropriate places.

\title{Test}

\author{}

\begin{document}

\listoftodos

\maketitle

\begin{abstract}
\end{abstract}

\section{Introduction}

\section{Background}
    \label{prior}
    \subsection{Supervised Learning}
        \subsubsection{Dataset and Model}
            We denote dataset as \(Y:= \{y_n = \left(  x_n , Z^n  \right)\} ^{N}_{n=1} \)
            where \(x \in \mathbb{R}^{d_x}\), \(z \in \mathbb{R}\)  for regression and \(z \in  \{ 0,1 \}\) for binary classification.
            学習の目的は\( y \approx F \left( x, \psi \right) \)を満たすparameter \(\psi \in \mathbb{R}^{p}\)を見つけることである。
            \(\psi\)はloss \(L \left( y, \psi \right) := L \left( F(x_n \psi) ,Z^{n} \right)\)の最小化問題の解であたえられる。
            \begin{equation}
                \psi = \mathop{\rm argmin}\limits_{ \psi }\frac{1}{N} \sum L \left( F(x_n;\psi) ,Z^{n} \right)
            \end{equation}
            ここで\(L\)は一般的な回帰問題ではl2 norm, 分類問題ではクロスエントロピーである。
            
        % SGD (Stochastic Gradient Descent)
          \subsubsection{SGD}
            ミニバッチを用いたSGDは,勾配を\( g \left( y, \psi ^{t} \right) := \nabla _{ \phi }L \left( y, \psi ^{t} \right) \),学習に使用されるサンプルのindexの集合を\(S_t \subset  \left\{ 1,  \ldots , N \right\}\)と表記すると,
            パラメータをステップ\(t\)においてに次のように更新する。    
            \begin{equation}
                \psi ^{t+1} 
                = \psi ^{t} 
                -\frac{\eta^{t}}{ \vert S_{t} \vert }  \sum _{i \in S_{t}}g_\psi\left( y_{i},\psi^{t}\right)
            \end{equation}

    \subsection{Prior Work}
        本研究の貢献の理解とcompletenessのため,Haraらの既存手法\cite{Hara2019}について説明する。Haraらは,SGDを用いる学習について,ある学習データのデータセットからの除外と再学習によるパラメータの変化を新たにSGD-Influenceとして定義した。Supervised Learningを対象に,SGD-Influenceのestimator, パラメータ変化に伴うlossの変化の推定に用いることができるLinear Influence Estimator (LIE) を提案した。
        \subsubsection{Counterfactual SGD}
            Counterfactual SGDは学習データからj番目のサンプルが除外されたデータセット\( Y\setminus\{y_j\}\)を用いた学習である。Counterfactual SGDのt番目のステップにおけるパラメータ\(\psi_{-j}^t\)の更新式は次のように与えられる。
            \begin{equation}
                \psi_{-j} ^{t+1} 
                = \psi_{-j} ^{t} 
                -\frac{\eta^{t}}{ \vert S_{t} \vert }  \sum _{i \in S_{t}\setminus\{j\}} g_\psi\left( y_{i},\psi_{-j}^{t}\right)
            \end{equation}
            なお,Counterfactual SGDにおける初期パラメータはSGDと等しい( \(\psi _{-j}^{1}= \psi ^{1} \) )ものとする。
        \subsubsection{SGD-Influence}
            
            \begin{equation}
              \Delta  \psi _{-j}^{t}:= \psi _{-j}^{t}- \psi ^{t}  
            \end{equation}
        
        \subsubsection{Approximation of Gradient Influence}
            我々は、勾配の一次テイラー近似を用いてSGD-influenceを推定する。損失関数\(L \left( y, \psi \right)\)が二階微分可能であると仮定し,Counterfactual SGDのパラメータ勾配\(g\left( y,\psi_{-j}^{t}\right)\)の\(\psi^t\)周りでの近似を求める。\(\Delta g_\psi^t\left( y \right) :=  g_{\psi} \left( y, \psi _{-j}^{t} \right) -g_{\psi} \left( y, \psi ^{t} \right)\)とすると,
            \begin{align}
                \Delta g_\psi^t\left( y \right) &\approx \nabla _{ \psi }^{2}L \left( y, \psi ^{t} \right) \Delta  \psi _{-j}^{t} \notag \\
                \leftrightarrow \frac{1}{ \vert S_{t} \vert } \sum _{i \in S_{t}} \{  \Delta g_\psi^t\left( y_i \right) &\approx H^{t} \Delta  \psi _{-j}^{t}
                \label{taylor}
            \end{align}
            \begin{equation*}
                \mathrm{where,~~}H^{t} := \frac{1}{ \vert S_{t} \vert } \sum _{i \in S_{t}}^{}\nabla _{ \psi }^{2}L \left( y_i, \psi ^{t} \right) 
            \end{equation*}
            
        \subsubsection{Estimator}
            \(\pi _{1} \left( j \right) ,  \ldots , \pi _{k} \left( j \right) , \ldots , \pi _{K} \left( j \right)\)を、\(K\)-epoch SGDでインスタンス\(y_j\)を使用するステップとする。合計ステップ\(T\)におけるSGD-Influenceは次のように与えられる。
         \begin{equation}
            \label{estimator}
            \Delta  \psi _{-j}^{T}\approx \sum _{k=1}^{K} \left(  \prod_{t=1}^{T}Z^{T-t} \right) \frac{\eta^{ \pi _{k} \left( j \right) }}{ \vert S_{ \pi _{k} \left( j \right) } \vert }g_\psi \left( y_{j}, \psi _{}^{ \pi _{k} \left( j \right) } \right)
        \end{equation}
        \begin{equation*}
            \mathrm{where,~~} Z^{t}:=I-\eta^{t}H^{t}  \notag
        \end{equation*}
        (\ref{estimator})の導出の詳細については,the estimatorのadversarial trainingへの適用可能性の検証に含まれる(\ref{applicability})。
            
        \subsubsection{Linear Influence Estimation with Estimator}
            Estimatorを用いてモデルのSGD-Influenceによる出力の変化を推定することができる。\(M\left(\psi^T\right)\)を合計ステップ\(T\)におけるパラメータ\(\psi^T\)をもつ任意の関数と表記し,SGD-Influenceによる出力の変化を\(\Delta M_{-j}^T := M \left(\psi _{-j}^{T} \right) -M \left( \psi ^{T} \right)\)と表記する。このとき次の線形近似が成り立つ。

            \begin{equation}
                \Delta M_{-j}^T \approx \nabla_\psi M \left( \psi ^{T} \right) ^\top \Delta \psi _{-j}^{T} \\
            \end{equation}
            
            \(u^{t}:=Z^{t}Z^{t+1} \cdots Z^{T-1} u^T \) where  \(u^T :=  \nabla_\psi M \left( \psi ^{T} \right) \) とすると,Estimatorをもちいて,
            \begin{equation}
                \label{lie}
                \Delta M_{-j}^T \approx \frac{1}{K} \sum _{k=1}^{K}\frac{\eta^{ \pi _{k} \left( j \right) }}{ \vert S_{\pi _{k} \left( j \right)}  \vert }{u^{ \pi _{k} \left( j \right) }}^\top g_\psi \left( y_{j}, \psi ^{ \pi _{k} \left( j \right) } \right)  \\
            \end{equation}


\bibliographystyle{named}
\bibliography{main}

\end{document}
