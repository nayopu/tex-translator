% \documentclass[xelatex,ja=standard,jafont=noto]{bxjsarticle}

%%%% ijcai20-multiauthor.tex

\typeout{Test}

% These are the instructions for authors for IJCAI-20.

\documentclass{article}

\usepackage[whole]{bxcjkjatype}
\pdfpagewidth=8.5in
\pdfpageheight=11in
% The file ijcai20.sty is NOT the same than previous years'
\usepackage{ijcai20}
% Use the postscript times font!
\usepackage{times}
\renewcommand*\ttdefault{txtt}
\usepackage{soul}
\usepackage{url}
\usepackage[hidelinks]{hyperref}
\usepackage[utf8]{inputenc}
\usepackage[small]{caption}
\usepackage{graphicx}
\usepackage{amsmath}
\usepackage{amssymb}
\usepackage{nccmath} 
\usepackage{bbm}
\usepackage{amsfonts}
\usepackage{booktabs}
\usepackage{todonotes}
\usepackage{algorithm}
\usepackage{algorithmic}
\urlstyle{same}

% the following package is optional:
%\usepackage{latexsym} 

% Following comment is from ijcai97-submit.tex:
% The preparation of these files was supported by Schlumberger Palo Alto
% Research, AT\&T Bell Laboratories, and Morgan Kaufmann Publishers.
% Shirley Jowell, of Morgan Kaufmann Publishers, and Peter F.
% Patel-Schneider, of AT\&T Bell Laboratories collaborated on their
% preparation.

% These instructions can be modified and used in other conferences as long
% as credit to the authors and supporting agencies is retained, this notice
% is not changed, and further modification or reuse is not restricted.
% Neither Shirley Jowell nor Peter F. Patel-Schneider can be listed as
% contacts for providing assistance without their prior permission.

% To use for other conferences, change references to files and the
% conference appropriate and use other authors, contacts, publishers, and
% organizations.
% Also change the deadline and address for returning papers and the length and
% page charge instructions.
% Put where the files are available in the appropriate places.

\title{Test}

\author{}

\begin{document}

\listoftodos

\maketitle

\begin{abstract}
\end{abstract}

\section{Introduction}

\section{Background}
    \label{prior}
    \subsection{Supervised Learning}
        \subsubsection{Dataset and Model}
            We denote dataset as \(Y:= \{y_n = \left(  x_n , Z^n  \right)\} ^{N}_{n=1} \)
            where \(x \in \mathbb{R}^{d_x}\), \(z \in \mathbb{R}\)  for regression and \(z \in  \{ 0,1 \}\) for binary classification.
            学習の目的は\( y \approx F \left( x, \psi \right) \)を満たすparameter \(\psi \in \mathbb{R}^{p}\)を見つけることである。
            \(\psi\)はloss \(L \left( y, \psi \right) := L \left( F(x_n \psi) ,Z^{n} \right)\)の最小化問題の解であたえられる。
            \begin{equation}
                \psi = \mathop{\rm argmin}\limits_{ \psi }\frac{1}{N} \sum L \left( F(x_n;\psi) ,Z^{n} \right)
            \end{equation}
            ここで\(L\)は一般的な回帰問題ではl2 norm, 分類問題ではクロスエントロピーである。
            
        % SGD (Stochastic Gradient Descent)
          \subsubsection{SGD}
            ミニバッチを用いたSGDは,勾配を\( g \left( y, \psi ^{t} \right) := \nabla _{ \phi }L \left( y, \psi ^{t} \right) \),学習に使用されるサンプルのindexの集合を\(S_t \subset  \left\{ 1,  \ldots , N \right\}\)と表記すると,
            パラメータをステップ\(t\)においてに次のように更新する。    
            \begin{equation}
                \psi ^{t+1} 
                = \psi ^{t} 
                -\frac{\eta^{t}}{ \vert S_{t} \vert }  \sum _{i \in S_{t}}g_\psi\left( y_{i},\psi^{t}\right)
            \end{equation}


\bibliographystyle{named}
\bibliography{main}

\end{document}
